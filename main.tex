\documentclass[12pt,a4paper,oneside]{report} %class

% hungarian language
\usepackage{t1enc}
\usepackage[magyar]{babel}
\selectlanguage{magyar}
\usepackage[utf8]{inputenc}

\usepackage{csquotes}

%\usepackage{float}
%\usepackage{tabularx}
%\usepackage[table]{xcolor}

% images 
\usepackage{graphicx}
\graphicspath{ {./images/} }

% set margins
\addtolength{\oddsidemargin}{-1.2cm}
\addtolength{\evensidemargin}{-1.2cm}
\addtolength{\textwidth}{2.4cm}
\addtolength{\topmargin}{-1.5cm}
\addtolength{\textheight}{3cm}

% set TOC depth
\setcounter{tocdepth}{4}

% remove red color for hyperref in TOC
\usepackage[unicode]{hyperref}
\hypersetup{
	colorlinks,
	citecolor=black,
	filecolor=black,
	linkcolor=black,
	urlcolor=black
}


% add bibtex
\usepackage[backend=biber]{biblatex}
\BiblatexHungarianWarningOff
\addbibresource{./mainbibliography.bib}



%top matter
\title{%
	\includegraphics[scale=0.1]{logo_oe}\\
	ÓBUDAI EGYETEM\\
	Neumann János Informatikai kar\\
	Mérnök informatikus BSc\\
	\vfill
	\large \textbf{Vizuális, interaktív programozás oktató rendszer moduláris megvalósítása\\}
	\large Projektmunka dokumentáció
	\vfill
}
\author{Ráncsik Áron}
\date{\today}

\begin{document}

\pagenumbering{Alph}
\begin{titlepage}
\maketitle
\thispagestyle{empty}
\end{titlepage}
\pagenumbering{arabic}


\newpage
\tableofcontents
\newpage


% Introductory chapters
\chapter{Bevezetés}
\par 
A legkülönbözőbb oktatásokban is manapság elterjedt\cite{riar2020game} gyakorlat az oktatott anyag játékos megfogalmazása, idegen szóval a \cite{Deterding2011} \textit{gamification}. A módszer lényege , hogy a megtanítani kívánt ismeretet nyers forma helyett játékos formában tesszük emészthetővé a tanulók számára. 
\par 
Dolgozatomban egy olyan programozás oktató rendszert dolgozok ki mely, a programozás oktatást igyekszik minél szélesebb korosztály számára elérhetővé és érhetővé tenni.
\section{Módszertan}
\subsection{Vizuális programozási környezet}
\subsection{Kiterjesztett valóság}
\subsection{Kinect szenzor}
\subsection{Gépi látás}
\subsubsection{Kamera mozgás becslés}
\subsubsection{Markerek keresése}
\subsection{Játék integráció}


\newpage
\chapter*{Irodalomjegyzék}
\addcontentsline{toc}{chapter}{Irodalomjegyzék}  
\printbibliography[heading=none]
\newpage
\listoffigures
\addcontentsline{toc}{chapter}{Ábrák jegyzéke}


\end{document}